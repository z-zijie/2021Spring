\documentclass{article}
\usepackage[utf8]{inputenc}
\usepackage[english]{babel}
\usepackage[]{amsthm}
\usepackage[]{amssymb}

\title{SP21 COMPSCI 513 - Homework 1}
\author{Zijie Zhang}
\date\today

\begin{document}
\maketitle

\begin{proof}
    \indent
    \begin{itemize}
        \item[(a)]
            By definition,
            $$\left\Vert Q \right\Vert_2 = max\left\{\frac{\left\Vert Qv \right\Vert_2}{\left\Vert v \right\Vert_2}:v\not=0\right\} = 1$$
            For $\left\Vert v \right\Vert_2 = 1$, $\left\Vert v \right\Vert_2^2 = 1$,
            $$\left\Vert Q \right\Vert_2^2 = 1 = \left\Vert v \right\Vert_2^2 = \left\Vert Qv \right\Vert_2^2 = (Qv, Qv) = (Q'Qv, v)$$
            $Q'Q$ is symmetric, by the schur decomposition, $Q'Q = SDS'$. Where $S$ is orthogonal and $D$ is diagonal.
            $$1 = (SDS'v, v) = (DS'v, S'v)$$
            Denote $w = S'v$, we have $(Dw, w) = 1$.
            $$1 = \sum D_{ii} w_i^2 = \sum w_i^2$$
            $$\sum \left(D_{ii} - 1\right)w_i^2 = 0$$
            So, $D = I$.
            1 is the only eigenvalue of $Q'Q$.
        \item[(b)] 
            From the proof of (a), $D = I$ means $$Q'Q = SS'$$
            Here $S$ is orthogonal, $Q$ must also be orthogonal.
    \end{itemize}
\end{proof}

\end{document}
